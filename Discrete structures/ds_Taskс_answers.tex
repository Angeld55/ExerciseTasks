\documentclass[]{article}
\usepackage[utf8]{inputenc}
\usepackage[bulgarian]{babel}

%opening
\title{Отговори на част от задачите по Дискретни структури}
\author{Ангел Димитриев}
\date{}
\begin{document}
	
	\maketitle
	
	
	\section{Логика}
	
	\subsection*{Задача 1} 
	
	Да. Съждението е противоречие. 
	
	
	\subsection*{Задача 3} 
	
	Стойността на съждението е лъжа.
	
	\subsection*{Задача 4} 
	
	Стойността на съждението е истина.
	
	
	\subsection*{Задача 5}
	
	Стойността на съждението е истина.
	
	
	\subsection*{Задача 6}
	
	Стойността на съждението е истина.
	
	
	\subsection*{Задача 8} 
	
	Двете множества са еквивалентни при тези условия.
	
	
	
	\subsection*{Задача 9} 
	
	Двете множества са еквивалентни при тези условия.
	Всещност това е множеството A.
	
	
	
	\subsection*{Задача 10}
	
	Двете множества $\textbf{не са}$ еквивалентни при тези условия.
	Елемент, които принадележи на множествата A и C, ще принадлежи на множество от лявата страна на равенството, но не на множеството от дясната страна на равенството.
	
	
	\subsection*{Задача 12}
	
	Броят на тези множества е  : $2^{n-m}$.
	
	\subsection*{Задача 13}
	
	Броят на тези множества е : $(2^n)+ (2^m) - 2$.
	
	\subsection*{Задача 14}
	
	Твърдението е вярно!
	
	
\end{document}


